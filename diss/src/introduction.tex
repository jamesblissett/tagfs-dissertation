\chapter{Introduction}
\vspace{1cm}

Organising files can be a tedious, error-prone, and unsatisfying activity. The
file system has survived to this day mostly unchanged from its initial origins
in modelling a real physical paper filing system \todo{ref here}. The
hierarchical structure enforced by the file system can feel confining and
inflexible. My project aims to attempt a different file management strategy,
whilst still maintaining compatibility and interoperability with the normal
hierarchical method. The file management strategy of my project is based upon
``tagging'' files (more specifically file paths) with small items of metadata
known as ``tags''. A tag can be either a simple string such as ``favourite'' or
it may consist of a tag name and a tag value for example ``year=2019''. These
tags may then be used as parameters in a search to find files that satisfy
certain criteria. Examples of searches could include, finding all the files
with the ``year=2019'' tag, but without the ``favourite'' tag.

The project maintains compatibility with the traditional hierarchical file
system structure by exposing a virtual file system using the FUSE technology
available for Linux. By using this file system, the user may perform arbitrary
queries against tagged files in the system by specifying their query as a file
path. They may also lookup a particular file by name to discover what tags are
associated with it. Using a virtual file system approach allows the user to
access their files using the tag based methodology from within all the normal
software applications and operating system components that they are familiar
with. The integration of the tag based strategy with the traditional
hierarchical method, is a much more feasible way of adopting tag based file
organisation compared to requiring the operating system and every user program
to be made aware of and updated to handle the new paradigm.

The intended users of the software are technically capable users who already
know their way around and are comfortable on the command line. The software
provides a command line interface alongside the virtual file system. This CLI
tool can be used to integrate the tag-based file system with the user's
existing CLI workflow and follows common CLI conventions to provide immediate
familiarity for the targeted technical users.

By using the project, the intended users will be able to manage their files in
a more flexible way, whilst also being able to attach extra metadata to a file
path. Additionally, the more powerful querying abilities of the software will
allow users to find their files more easily, and may even allow the user to
be able to identify common properties between files they previously believed
unrelated.

Whilst developing the project, I made use of several software development
techniques such as automated testing, creating documentation, and static
analysis. In later sections I will justify their usage and evaluate how they
helped me to develop a robust and maintainable system.

\vspace{4mm}
The specific aims for my project are as follows:
\begin{itemize}
    \item Discuss and justify the use of particular appropriate technologies
        for building the software.
    \item Design and implement a queryable tag-based file system using the FUSE
        technology by following professional software development strategies.
    \item Evaluate how using professional software development techniques
        contributed to the success of the project.
    \item Explore potential improvements that could be made to the project
        itself and to the development techniques used to create the software.
\end{itemize}
