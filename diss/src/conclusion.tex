\chapter{Conclusion}
\vspace{1cm}

I believe that my project has been successful in achieving its goals to create
a tag based file system. This can be seen to be the case through the
documentation given in Section~\ref{section:features} regarding the features of
the CLI tool and the virtual file system. Additionally, the code that
implements these features can be found in the \verb@src/@ directory of the
accompanying code repository. The definition of the CLI interface is located in
the \texttt{cli.rs} file and the entry points for each subcommand can be found
in the \texttt{tagfs.rs} file. The code for the file system resides within the
\texttt{lib/fs.rs} and \texttt{lib/fs/entries.rs} files. The database layer can
be found in the \texttt{lib/db.rs} file and its sub-modules can be found in the
\texttt{lib/db/} directory. These include the query builder and the edit
subcommand representation format. Combining these components together results
in a reasonably useful package for managing files using a tag based strategy
and accessing them from a virtual file system.

The project has been developed using good professional practices. These
practices include automated testing (both unit and integration style), modular
design, static analysis, and creating documentation. I have discussed these
practices and how they have helped me to develop \texttt{tagfs} in more detail
in Section~\ref{sec:implementation}. Evidence for these practices can be found
throughout the project code, such as documentation comments on public functions
and comments used throughout the code to explain particularly tricky or
confusing sections of code. Additionally, evidence for testing can be found in
the \texttt{tests/} directory in the root of the project. These are the
integration tests. Unit tests are located in the main project directory (as per
Rust testing guidelines found in the Rust book \cite{rust-testing}). An example
set of unit tests can be found in \texttt{src/lib/db/query/tests.rs}.

The project was developed with a specific group of users in mind, namely
technical users on Linux. I believe that I have produced a piece of software
that is tailored for these users and reflects their needs.
This is discussed in both the design period of the project in
Section~\ref{sec:design} and in more detail in the evaluation in
Chapter~\ref{chap:evaluation}. I believe the software is useful for these
technical users because it integrates with their existing work flow. This is
achieved via exposing the tag database via the file system, and via the usage
of the convention following command line tool. These compose together to
produce a familiar and easy to grasp experience for the target user base.
