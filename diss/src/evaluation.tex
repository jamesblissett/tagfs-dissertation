\chapter{Evaluation}
\label{chap:evaluation}
\vspace{1cm}

I believe that the \texttt{tagfs} software is useful and achieves its goals of
implementing a method of tag based file management via a virtual file system.
This is realised by the file system component of \texttt{tagfs} that allows the
user to browse and query their files in a more abstract way by using tags
instead of file system paths. The additional task of tag management takes place
using the command line tool. Together these allow a user to have a reasonably
complete tag-based file management experience. The implementation of the
virtual file system of \texttt{tagfs} is robust and handles edge cases
gracefully without unintended crashes. This is achieved through extensive use
of Rust's ``Result'' type to easily propagate errors higher up the call stack
to where they are more relevant, and via attaching context messages to lower
level error messages to make them more user friendly for both the programmer
and the user. Using the Rust ``Result'' type also makes thinking about error
handling mandatory, because to access a value wrapped in a result type you must
handle the error or explicitly ignore the error. With the use of this pattern,
it is impossible to forget to handle an error as all error handling is
explicit. This is in contrast to languages such as C where error handling is
often handled through negative return values (this is the case with the fork()
libc function) this makes it very easy to forget to handle error cases and as a
result can reduce the robustness of the software.

Another way that I found Rust to have been a good fit for my project was via
the use of the Rust build system \texttt{cargo}. The cohesiveness of
\texttt{cargo} in containing a documentation generator, testing harness and
static analysis tool all in one package, meant that I could more easily focus
on developing the project correctly rather than fighting with tooling. The fact
that I did not have to choose a testing library and configure it, allowed me to
just get on with writing tests. This meant that I was more likely to write more
tests as there was such a low barrier to entry for creating them.

The project's robustness is further improved by ``standing on the shoulders of
giants''. In the case of \texttt{tagfs} this means the use of high quality
libraries, and development tools. Both of the two main libraries used
(\texttt{fuser} and \texttt{rusqlite}) proved to be as high quality as I had
hoped them to be. However, the documentation for the \texttt{fuser} was a
little lacking, but I was able to remedy this by reading the source code.
Thankfully the code seemed to be written with care, and I was able to navigate
the foreign code base successfully to solve my issues. The \texttt{rusqlite}
library had no issues with documentation, and I could also use most of the
existing documentation for SQLite when it was insufficient. I believe that
using an SQL database proved to be a good decision for this project as it was
very easy to integrate with the code that I had already created up until that
point. This was compounded by the fact that I was already familiar with the
basics of SQL and I had experience with SQLite in other projects.

The design of the system is flexible and does not restrict the user or funnel
them towards any particular tagging methodology. This allows the user to create
their own tagging system backed by \texttt{tagfs} rather than forcing a
structure onto them. This flexibility is especially appreciated by the target
user base, because (as power users) they do not want to be artificially
constrained within a system that limits them to only the ideas that the
developer originally thought of. The idea of flexibility is also extended to
the command line component of \texttt{tagfs}, which is specifically designed to
allow extensibility via the traditional UNIX method of pipes and shell scripts.
This again allows the user to mould the system to their preference rather than
relying on the developer to have conceived of every use case beforehand.
Sticking to established platform conventions for the CLI, allows the user to
easy integrate \texttt{tagfs} into their existing command line based workflow,
and allows the software to feel more natural to existing technical users.
Additionally, the file system component of \texttt{tagfs} naturally fits into a
technical UNIX user's environment by virtue of the old adage ``\emph{everything
is a file}''. This further strengthens the integration between \texttt{tagfs}
and the user's everyday tools by allowing them to interact with \texttt{tagfs}
like any other part of the system. The last way in which \texttt{tagfs}
integrates with the user's system is through the \texttt{edit} subcommand. This
allows the user to edit the tag database with their preferred editor. I believe
this is a much better solution for editing the tag database than developing a
full custom editor. This is because it allows the user extreme flexibility in
how they edit the tag database, and it reuses the user's editor to do so, which
as a power user they are surely familiar and comfortable with. This allows the
user to forgo learning a single use specific editor, and instead focus on the
task of editing the tag database.

Using a DSL (domain specific language) for querying the tag database, is
another way in which the system delivers the flexibility required by the target
user base. The language allows users to be specific about defining a query that
returns exactly the results they are interested in. This in combination with
the stored queries feature, allows the user to spend the time to set up queries
exactly as they desire, and then save them for later easy access. Setting up
these stored queries and learning the query language itself does require
initial investment into the software, but I believe that this is not a problem
for the targeted users. This is because, as power users, they are used to
software requiring initial effort to learn, but ultimately providing lots of
value once over an initial hurdle. Examples of software in this category are
the traditional UNIX power user utilities such as vim, emacs and the shell.

I believe \texttt{tagfs} to be an example of a reasonably professionally
developed software project. The usage of specific professional development
techniques are spoken about in detail in Section~\ref{sec:implementation}. In
general, the use of professional practices greatly helped me throughout the
project. An example of this is building confidence in my code via testing,
which allowed me to feel comfortable in modifying and extending my code,
because I knew that any accidental modification to existing functionality would
be caught by the tests. The use of static analysis tools helped me to write
clearer and more correct code by warning me about potential problems with the
code base. Keeping on top of these potential problems resulted in the code base
being easier to read and therefore reason about. This could also have
contributed to the robustness of the project through improving the clarity of
the code and therefore making bugs easier to find.

Using modular design principles such as encapsulation allowed me to more easily
reason about my code, because it allowed me to hide implementation details that
distract from the higher level data flow. This results in code that reads more
nicely as a list of actions to perform rather than an unnecessarily complex
information overload. An example of this in the code is the database module.
Encapsulating all database functionality within a dedicated module, allowed me
to expose a simple interface to the rest of the program for working with the
database. This means that I did not have to give direct access to the
underlying database connection to other parts of the application. As a
consequence of this, I can rely on the program only performing known actions
via the interface, so there is no ad hoc SQL scattered around the program. The
increased code clarity provided by this sort of modular design results in more
maintainable code which, for \texttt{tagfs} means that in the future someone
new to the codebase could implement some of the features described in
Section~\ref{sec:potential-features} with a minimal onboarding period as they
work to understand the project's code base. More maintainable code is not just
a benefit for other developers, as when coming back to your own code after a
long period of time it can be difficult to remember why certain design
decisions were made. This is where documenting and making the code as obvious
as possible is important, for which modular design is a great helper.

\section{Potential Future Work}
\label{sec:potential-features}

\todo{Tag dependency or grouping - eg a 'film' tag implies that you must have
an 'imdb' tag also.}

Due to the modular design and iterative development style used to develop the
\texttt{tagfs} project, it would be easy to extend the project with new
features and even entirely new interfaces. For example, it could be interesting
to create a GUI for \texttt{tagfs}. This would allow users to browse and edit
the tag database in a more visual way. A more visual style of tag manipulation
could broaden the intended user base considerably. This is because allowing the
users to modify the tag database through a GUI would remove the requirement for
a user to have experience with the command line, and since almost all
non-technical users are inexperienced with the command line this could
potentially create a more feasible route to them using \texttt{tagfs}. For
technical users however, I do not think a GUI would bring a great deal of
value. This is because I think that most technical users would want to automate
their interactions with \texttt{tagfs} via scripting, and only rarely would
they need to manually interact with the program. In this case of rare manual
interaction, it would not make sense to open a dedicated GUI and attempt to
remember its idiosyncrasies. I also believe technical users would prefer to
interact via the CLI because it would be more familiar to them after having
development scripts around it.

Another potential avenue that the project's development could be taken down is
allowing the user to modify the tag database from the file system. This would
make using the CLI tool less necessary, and make editing the tag database more
intuitive and natural. There are lots of ways a feature like this could work.
One way would be to allow the user to remove a tag directory within the virtual
file system, the file system would then intercept the request and as a
consequence remove the tag represented by the deleted directory from all file
paths that it is currently tagged on. Another way could be to allow the editing
of a file's tags by editing the ``.tags'' file that is automatically generated
by the file system. This could work similarly to the edit subcommand from the
CLI. Even deleting a ``.tags'' file could be implemented to remove all traces
of that file path from the tag database.

One feature that could be useful for users would be the ability to customise
the autotagging process. This could be done with custom user created rules that
use regular expressions to match path components or specific file metadata.
After a rule has been matched, an action can take place to generate tags for
the matched file. This could be as simple as assigning a static string to a
given rule or as complex as allowing an external script to be used to generate
a set of tags. This would allow the user to be more confident in using the
autotagging feature as they would be sure it would tag files exactly to their
specifications, and an increase in the usage of autotagging means there is less
friction in adding new files to the system.
