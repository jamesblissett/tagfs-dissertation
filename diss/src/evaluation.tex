\chapter{Evaluation}
\vspace{1cm}

\todo{Probably this is better in another section I think? Talk about who the
users are and why it is appropriate for them here?}

\todo{Talk about potential future improvements and why they would be
improvements for the intended userbase}

\section{Potential Future Work}

Due to the modular design and iterative development style used to develop the
\texttt{tagfs} project, it would be easy to extend the project with new
features and even entirely new interfaces. For example, it could be interesting
to create a GUI for \texttt{tagfs}. This would allow users to browse and edit
the tag database in a more visual way. A more visual style of tag manipulation
could broaden the intended user base considerably. This is because allowing the
users to modify the tag database through a GUI would remove the requirement for
a user to have experience with the command line, and since almost all
non-technical users are inexperienced with the command line this could
potentially create a more feasible route to them using \texttt{tagfs}. For
technical users however, I do not think a GUI would bring a great deal of
value. This is because I think that most technical users would want to automate
their interactions with \texttt{tagfs} via scripting, and only rarely would
they need to manually interact with the program. In this case of rare manual
interaction, it would not make sense to open a dedicated GUI and attempt to
remember its idiosyncrasies. I also believe technical users would prefer to
interact via the CLI because it would be more familiar to them after having
development scripts around it.

Another potential avenue that the project's development could be taken down is
allowing the user to modify the tag database from the file system. This would
make using the CLI tool less necessary, and make editing the tag database more
intuitive and natural. There are lots of ways a feature like this could work.
One way would be to allow the user to remove a tag directory within the virtual
file system, the file system would then intercept the request and as a
consequence remove the tag represented by the deleted directory from all file
paths that it is currently tagged on. Another way could be to allow the editing
of a file's tags by editing the ``.tags'' file that is automatically generated
by the file system. This could work similarly to the edit subcommand from the
CLI. Even deleting a ``.tags'' file could be implemented to remove all traces
of that file path from the tag database.

One feature that could be useful for users would be the ability to customise
the autotagging process. This could be done with custom user created rules that
use regular expressions to match path components or specific file metadata.
After a rule has been matched, an action can take place to generate tags for
the matched file. This could be as simple as assigning a static string to a
given rule or as complex as allowing an external script to be used to generate
a set of tags. This would allow the user to be more confident in using the
autotagging feature as they would be sure it would tag files exactly to their
specifications, and an increase in the usage of autotagging means there is less
friction in adding new files to the system.
